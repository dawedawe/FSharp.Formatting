\documentclass{article}

\usepackage{xcolor}
\usepackage{graphicx}
\usepackage{fancyvrb}
\usepackage{listings}
\usepackage[T1]{fontenc}
\usepackage{hyperref}
\usepackage{amsmath}

\definecolor{officegreen}{rgb}{0, 0.5, 0}
\definecolor{navy}{rgb}{0, 0, 0.5}
\definecolor{linecolor}{rgb}{0.5, 0.6875, 0.6875}
\definecolor{outputcolor}{rgb}{0.375, 0.375, 0.375}

\newcommand{\id}[1]{\textcolor{black}{#1}}
\newcommand{\com}[1]{\textcolor{officegreen}{#1}}
\newcommand{\inact}[1]{\textcolor{gray}{#1}}
\newcommand{\kwd}[1]{\textcolor{navy}{#1}}
\newcommand{\num}[1]{\textcolor{officegreen}{#1}}
\newcommand{\ops}[1]{\textcolor{purple}{#1}}
\newcommand{\prep}[1]{\textcolor{purple}{#1}}
\newcommand{\str}[1]{\textcolor{olive}{#1}}
\newcommand{\lines}[1]{\textcolor{linecolor}{#1}}
\newcommand{\fsi}[1]{\textcolor{outputcolor}{#1}}
\newcommand{\omi}[1]{\textcolor{gray}{#1}}

% Overriding color and style of line numbers
\renewcommand{\theFancyVerbLine}{
\lines{\small \arabic{FancyVerbLine}:}}

\lstset{%
  backgroundcolor=\color{gray!15},
  basicstyle=\ttfamily,
  breaklines=true,
  columns=fullflexible
}

\title{{page-title}}
\date{}

\begin{document}

\maketitle


\section*{Command line}



To use F\# Formatting tools via the command line, you can use the \texttt{fsdocs} dotnet tool.
\begin{lstlisting}
dotnet tool install fsdocs-tool
dotnet fsdocs [command] [options]

\end{lstlisting}
\subsection*{The build command}



This command processes a \texttt{docs} directory and generates API docs for projects in the solution according to the
rules of \href{apidocs.html}{API doc generation}. The input accepted is described in \href{content.html}{content}.
\begin{lstlisting}
fsdocs build

\end{lstlisting}


The command line options accepted are:
\begin{tabular}{|l|l|}\hline
\textbf{Command Line Option} & \textbf{Description}\\ \hline\hline
\texttt{--input} & Input directory of content (default: \texttt{docs})\\ \hline
\texttt{--projects} & Project files to build API docs for outputs, defaults to all packable projects\\ \hline
\texttt{--output} & Output Directory (default 'output' for 'build' and 'tmp/watch' for 'watch')\\ \hline
\texttt{--noapidocs} & Disable generation of API docs\\ \hline
\texttt{--ignoreprojects} & Disable project cracking\\ \hline
\texttt{--eval} & Evaluate F\# fragments in scripts\\ \hline
\texttt{--saveimages} & Save images referenced in docs\\ \hline
\texttt{--nolinenumbers} & Don't add line numbers, default is to add line number.\\ \hline
\texttt{--parameters} & Additional substitution parameters for templates\\ \hline
\texttt{--nonpublic} & The tool will also generate documentation for non-public members\\ \hline
\texttt{--nodefaultcontent} & Do not copy default content styles, javascript or use default templates\\ \hline
\texttt{--clean} & Clean the output directory\\ \hline
\texttt{--help} & Display this help screen\\ \hline
\texttt{--version} & Display version information\\ \hline
\texttt{--properties} & Provide properties to dotnet msbuild, e.g. --properties Configuration=Release Version=3.4\\ \hline
\texttt{--fscoptions} & Additional arguments passed down as \texttt{otherflags} to the F\# compiler when the API is being generated.<br/>Note that these arguments are trimmed, this is to overcome \href{https://github.com/commandlineparser/commandline/issues/58}{a limitation in the command line argument processing}.<br/>A typical use-case would be to pass an addition assembly reference.<br/>Example \texttt{--fscoptions " -r:MyAssembly.dll"}\\ \hline
\texttt{--strict} & Fail if docs are missing or can't be generated\\ \hline
\end{tabular}



The following command line options are also accepted but it is instead recommended you use
settings in your .fsproj project files:
\begin{tabular}{|l|l|}\hline
\textbf{Command Line Option} & \textbf{Description}\\ \hline\hline
\texttt{--sourcefolder} & Source folder at time of component build (\texttt{<FsDocsSourceFolder>})\\ \hline
\texttt{--sourcerepo} & Source repository for github links (\texttt{<FsDocsSourceRepository>})\\ \hline
\texttt{--mdcomments} & Assume comments in F\# code are markdown (\texttt{<UsesMarkdownComments>})\\ \hline
\end{tabular}



The command will report on any \texttt{.fsproj} files that it finds, telling you if it decides to skip a particular file and why.


For example, a project will be skipped if:
\begin{itemize}
\item 

The project name contains ".Tests" or "test" (because it looks like a test project)

\item 

The project does not contain
\begin{lstlisting}[numbers=left]

[escapeinside=\\\{\}]
{<}\id{GenerateDocumentationFile}{>}\kwd{true}\ops{</}\id{GenerateDocumentationFile}{>}


\end{lstlisting}


\end{itemize}

\subsection*{The watch command}



This command does the same as \texttt{fsdocs build} but in "watch" mode, waiting for changes. Only the files in the input
directory (e.g. \texttt{docs}) are watched. A browser will be launched automatically (unless \texttt{--nolaunch} is specified).


You will need to ensure that the input directory exists, and contains at least \texttt{index.md}, otherwise the browser will
report an error (e.g. "Problem loading...", "Connection was reset").
\begin{lstlisting}
fsdocs watch

\end{lstlisting}


Restarting may be necesssary on changes to project files. The same parameters are accepted, plus these:
\begin{tabular}{|l|l|}\hline
\textbf{Command Line Option} & \textbf{Description}\\ \hline\hline
\texttt{--noserver} & Do not serve content when watching.\\ \hline
\texttt{--nolaunch} & Do not launch a browser window.\\ \hline
\texttt{--open} & URL extension to launch \href{http://localhost:<port>/\%s.}{http://localhost:<port>/\%s.}\\ \hline
\texttt{--port} & Port to serve content for \href{http://localhost}{http://localhost} serving.\\ \hline
\end{tabular}

\subsection*{Searchable docs}



When using the command-line tool a Lunr search index is automatically generated in \texttt{index.json}.


A search box is included in the default template.  To add a search box
to your own \texttt{\_template.html}, include \texttt{fsdocs-search.js}, which is added to the \texttt{content}
by default.
\begin{lstlisting}
...
<div id="header">
  <div class="searchbox">
    <label for="search-by">
      <i class="fas fa-search"></i>
    </label>
    <input data-search-input="" id="search-by" type="search" placeholder="Search..." />
    <span data-search-clear="">
      <i class="fas fa-times"></i>
    </span>
  </div>
</div>
...

\end{lstlisting}


\end{document}